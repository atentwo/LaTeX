\documentclass{article}
\usepackage{aten}

\newcommand{\documenttitle}{aten.sty}
\newcommand{\authorname}{atentwo}

\begin{document}

\title{\documenttitle}
\rhead{\textbf{\small \documenttitle}}
\lhead{\textbf{\small \authorname}}
\author{\authorname}
\maketitle

\tableofcontents 

\newpage

\section{Introduction}

\texttt{aten.sty} provides a simple and lightweight solution to make your LaTeX documents prettier.

\subsection{How to Use}

Prepend the following to your tex file:
\begin{verbatim}
\documentclass{article}
\usepackage{aten}

\newcommand{\documenttitle}{DOCUMENT TITLE}
\newcommand{\authorname}{AUTHOR NAME}

\begin{document}

\title{\documenttitle}
\rhead{\textbf{\small \documenttitle}}
\lhead{\textbf{\small \authorname}}
\author{\authorname}
\maketitle

\tableofcontents
\newpage

% Content goes here

\end{document}
\end{verbatim}

\subsection{Environments}

These are what the environments look like.
\begin{definition}
This is a definition.
\end{definition}

\begin{theorem} \label{thm:oddposintdivby5}
For odd, positive integers $k$, $1 + 4^k$ is divisible by 5. 
\end{theorem}

\begin{proof}
We prove that \( 1 + 4^k \equiv 0 \pmod{5} \) for all odd, positive integers \( k \).
Note that \( 4 \equiv -1 \pmod{5} \)
\[
4^k \equiv (-1)^k \pmod{5}.
\]
For odd \( k \), \( (-1)^k = -1 \), so:
\[
4^k \equiv -1 \pmod{5} \Rightarrow 1 + 4^k \equiv 1 + (-1) = 0 \pmod{5}.
\]
Hence, \( 1 + 4^k \) is divisible by 5 for all odd \( k \in \mathbb{Z}^+ \).
\end{proof}
\begin{remark}
This is a remark.
\end{remark}

\begin{example} 
This is an example.

This example has more than one line.
\end{example}

\subsection{More Features}

Links are supported. \ref{thm:oddposintdivby5} can either be proven through induction on the odd positive integers, or through a modular arithmetic argument.

\section{Conclusion}

Last updated 2025-05-22.

\end{document}
